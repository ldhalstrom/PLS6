\documentclass{article}
\usepackage[paperwidth=.5\paperwidth,paperheight=.25\paperheight]{geometry}
\usepackage{pgfpages}
\pagestyle{empty}
\thispagestyle{empty}
\pgfpagesuselayout{8 on 1}[a4paper]
\makeatletter
\@tempcnta=1\relax
\loop\ifnum\@tempcnta<9\relax
\pgf@pset{\the\@tempcnta}{bordercode}{\pgfusepath{stroke}}
\advance\@tempcnta by 1\relax
\repeat
\makeatother

%%%%%%%%%%%%%%%%%%%%%%%%%%%%%%%%
\newenvironment{flashcard}[2][]{%
\noindent  \textsc{#1} %flash card title text

\vfill
\centerline{{\Large\emph{#2}}} %flashcard word text
\vfill
\newpage
}
{\newpage}
%%%%%%%%%%%%%%%%%%%%%%%%%%%%%%%%

\usepackage[latin1]{inputenc}
\usepackage{amsfonts}
\usepackage{amsmath}

\begin{document}



%%%%%%%%%%%%%%%%%%%%%%%%%%%%%%%%%%%%%%%%%%%%%%%%%%%%%%%%%%%%%%%%%%%%%%%%
%%% NOTES ON HOW TO MAKE FLASHCARDS
%%%%%%%%%%%%%%%%%%%%%%%%%%%%%%%%%%%%%%%%%%%%%%%%%%%%%%%%%%%%%%%%%%%%%%%%
    %read in all .jpg, .png, and .JPG files
    %parse filenames from extensions
    %Layout 8 pictures on page grid
    %and 8 corresponding filenames on next page in alternated grid



% \begin{python}
% \end{python}






\begin{flashcard}[Definition]{Method of Frobenius}

\vspace*{\stretch{1}}
\begin{displaymath}
w(x,r)=(x-x_{0})\sum_{n=0}^{\infty}a_{n}(x-x_{0})^{n}
\end{displaymath}
\vspace*{\stretch{1}}
\end{flashcard}

\begin{flashcard}[Definition]{Hamiltonian operator}
\vspace*{\stretch{1}}
\begin{displaymath}
\textrm{\^H} = -\frac{\hbar^2}{2m}\nabla^2 + V
\end{displaymath}
\vspace*{\stretch{1}}
\end{flashcard}


%manual flash card without function
\noindent  test title label
\vfill
\centerline{{\Large\emph{test title text}}} %flashcard word text
\vfill
\newpage

\vspace*{\stretch{1}}
\begin{center}
test picture
\end{center}
\vspace*{\stretch{1}}
\newpage






\end{document}
